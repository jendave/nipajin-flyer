% (c) 2009-2016 by Markus Leupold-Löwenthal
% This file is released under CC BY-SA 4.0. Please do not apply one-way compatible licenses.

\renewcommand{\flyerVersion}{v0.1}

% CHANGELOG-de
%
% 0.1
%   - Erstfassung

% --- language dependent typography stuff --------------------------------------

\renewcommand{\say}[1]{„\textit{#1}“}
\setdefaultlanguage[spelling=new]{german}

\renewcommand{\fsNormal}{\fontsize{11pt}{13.2pt plus 0pt minus 0pt}}
%\renewcommand{\fsSmall}{\fontsize{9pt}{10pt plus 0.1pt minus 0pt}}

% --- pdf metadata -------------------------------------------------------------

\hypersetup{
	pdfsubject={Ein NIP'AJIN Rollenspiel auf einem Flyer. Setting und Regeln inklusive.},
}

% --- front page ---------------------------------------------------------------

\renewcommand{\rulesTitle}{

	\noindent
	\color{primary}\setbox\formbox\hbox{\addfontfeature{LetterSpace=-3.0}GESTATTEN, }\usebox\formbox
	\color{white}\makebox[\dimexpr\linewidth-\wd\formbox\relax][l]{\fffancy\expandafter\spaceout{FLYER}} \par
	\noindent
	\makebox[\linewidth][l]{\fffancy\spaceout{ROLLENSPIEL!}} \par

}

\renewcommand{\rulesHeadline}{Spielregeln}

\renewcommand{\rulesText}{%

	\noindent
	Dieser Flyer enthält eine Kurzfassung von \keyword{\nipajin}, einem kostenlosen, universellen Rollenspielsystem.

	Dein \keyword{Charakter} benötigt je einen W4, W6, W8, W10 und W12. Einen davon ernennst du zum \keyword{Widerstandswürfel} \HD. Lege\ ihn mit seiner höchsten Zahl auf den Tisch. Sinkt er im Spiel unter Eins, ist dein Charakter \keyword{überwunden} und scheidet aus. Die verbleibenden vier Würfel sind deine \keyword{verfügbaren Würfel}.

	Stellt sich die Frage, ob deinem Charakter eine Aktion gelingt, wählst du einen \keyword{Aktionswürfel} \AD\ aus deinen verfügbaren Würfeln und würfelst diesen. Wehrt sich niemand, ist die Aktion bei einer \keyword{Vier oder mehr} ein Erfolg.

	Soll deine Aktion einem Gegner schaden, muss dessen Spieler zur Verteidigung einen \keyword{Reaktionswürfel} \RD\ wählen. Nur wenn dein \AD\ einen höheren Wert als der \RD\ erwürfelt (unabhängig von einer Vier oder mehr), war dein Angriff ein Erfolg. Der \HD\ deines Gegners sinkt dann um einen Punkt.

	Nach einem Wurf ist der benutzte Würfel \keyword{verbraucht} und du legst ihn beiseite. Erst wenn du alle deine Würfel verbraucht hast, werden sie wieder verfügbar.

	\hfill Spiele jetzt das Szenario auf der anderen Seite!

}

\renewcommand{\rulesURL}{LUDUS-LEONIS.COM}

% --- back page ----------------------------------------------------------------

\renewcommand{\settingTitle}{

	\noindent
	\color{primary}\setbox\formbox\hbox{\addfontfeature{LetterSpace=-3.0}GESTATTEN, }\usebox\formbox
	\color{white}\makebox[\dimexpr\linewidth-\wd\formbox\relax][l]{\fffancy\expandafter\spaceout{FLYER}} \par
	\noindent
	\makebox[\linewidth][l]{\fffancy\spaceout{ROLLENSPIEL!}} \par

}

\renewcommand{\settingHeadline}{Szenario: Kleine Fische}

\renewcommand{\settingText}{%
	\noindent
	Dieses Szenario für 2 bis 6 Charaktere nutzt die Regeln von der anderen Seite.

	Einigt euch auf eine Stadt, einen kriminellen Hintergrund und eine Zeit, etwa die Mafia in New York um 1950 oder die Yakuza im heutigen Osaka. Die Charaktere sind dort konkurrierende \keyword{Kleinkriminelle}. Jeder Charakter erhält eine Beschreibung und einen Namen. Benutzt den W4 als \HD. Ziel jedes Charakters ist es, die anderen zu überwinden, um selbst in die Gunst vom \keyword{Big Boss} -- dem Spielleiter -- zu kommen.

	Das Spiel wird in \keyword{Runden} von etwa einer Woche abgehandelt. Der Big Boss stellt der Gruppe jede Woche eine Aufgabe (Juwelenraub, Entführung,~\ldots), die ausgespielt wird, deren Erfolg aber nicht so wichtig ist. Innerhalb dieser Aufgabe darf jeder Charakter jede Woche einmalig versuchen, einem anderen zu schaden und dessen \HD\ um 1 zu senken. Überwundene Charaktere verlassen am Ende der Woche frustriert die Stadt.

	Ein Mal pro Spiel darf jeder, der sich ungerecht behandelt fühlt, zum Big Boss gehen und \keyword{um einen Gefallen bitten}. Gelingt der nötige Wurf (4+), lässt er alle verprügeln, die diese Woche den Charakter angegriffen haben (automatisch -2  am \HD). Misslingt der Wurf, wird jedoch der Charakter selbst bestraft und verprügelt.

}

\renewcommand{\settingURL}{LUDUS-LEONIS.COM}
