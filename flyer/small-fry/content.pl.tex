% (c) 2009-2016 by Markus Leupold-Löwenthal
% This file is released under CC BY-SA 4.0. Please do not apply one-way compatible licenses.

% CHANGELOG-pl
%
% v1.1
%   - migrated to LaTeX
% v1.0
%   - translation by Szymon Piecha

% --- language dependent typography stuff --------------------------------------

\renewcommand{\say}[1]{„\textit{#1}”}
\setdefaultlanguage{polish}

\renewcommand{\fsFront}{\fontsize{12pt}{13.9pt plus 0pt minus 0pt}}
\renewcommand{\fsBack}{\fontsize{11.5pt}{13.9pt plus 0pt minus 0pt}}

% --- pdf metadata -------------------------------------------------------------

\hypersetup{
	pdfsubject={Nieskomplikowany, uniwersalny system RPG.},
	pdfkeywords={nipajin, nip'ajin, system, uniwersalny, RPG}
}

\renewcommand{\flyerVersion}{v1.1.0}
\renewcommand{\flyerCredits}{%
	Tłumaczenie:~Szymon Piecha; Layout:~Markus Leupold-Löwenthal; Ilustracje:~Bettina Ott, all-silhouettes.com%
}

% --- back page ----------------------------------------------------------------

\renewcommand{\settingTitle}{

	\noindent\color{primary}\fontsize{14pt}{12mm}\selectfont\addfontfeature{WordSpace=2,LetterSpace=6}
	„Mam dla ciebie ulotkę nie do wyrzucenia.”

	\noindent\color{white}\fffancy\fontsize{10.5mm}{12mm}\selectfont\addfontfeature{WordSpace=2,LetterSpace=2}
	ERPEGOWA ULOTKA

}

\renewcommand{\settingHeadline}{Scenariusz: Płotki}

\renewcommand{\settingText}{%
	\noindent
	Scenariusz ten jest przeznaczony dla 2 do 6 osób i korzysta z reguł na drugiej stronie.

	Wybierzcie grupę przestępczą i związane z nią miasto i okres działalności np. mafia Nowego Jorku z 1950 roku albo Yakuza z dzisiejszej Osaki. Postacie odgrywają konkurujących ze sobą \keyword{bandziorów}. Każda Postać posiada opis oraz imię. Używajcie k4 jako \HD. Celem każdej Postaci jest pokonanie reszty, by otrzymać względy \keyword{„big bossa”}, którego odgrywa prowadzący.

	Rozgrywka jest podzielona na \keyword{rundy} trwające jeden tydzień. W każdej rundzie big boss zleca wam jakieś zadanie (kradzież klejnotów, uprowadzenie...), lecz to nie wykonanie zadania powinno być dla was najważniejsze. Raz podczas każdej rundy każdy z was może zaszkodzić innej Postaci i dzięki temu obniżyć jej \HD\ o 1. Pokonane postacie pod koniec tygodnia opuszczają miasto z lękiem patrząc przez ramię.

	Jeżeli gracz uzna, że jest nieuczciwie traktowany przez pozo-stałych członków grupy, może \keyword{poprosić big bossa o przysługę}. Jeśli Postaci uda się rzut (4+), boss obija mordę wszystkim osobom, które zaatakowały poszkodowanego (automatycznie obniża to ich \HD\ o 2). Jeśli jednak skarżącemu nie uda się rzut, to on zostanie pobity.

}

\renewcommand{\settingURL}{\href{https://ludus-leonis.com/pl}{ludus-leonis.com/pl}}
