% (c) 2009-2016 by Markus Leupold-Löwenthal
% This file is released under CC BY-SA 4.0. Please do not apply one-way compatible licenses.

% CHANGELOG-de
%
% v1.1
%   - migrated to LaTeX
% v1.0
%   - first version

% --- language dependent typography stuff --------------------------------------

\renewcommand{\say}[1]{„\textit{#1}“}
\setdefaultlanguage[spelling=new]{german}

\renewcommand{\fsFront}{\fontsize{11pt}{13.2pt plus 0pt minus 0pt}}
\renewcommand{\fsBack}{\fontsize{11pt}{13.2pt plus 0pt minus 0pt}}

% --- pdf metadata -------------------------------------------------------------

\hypersetup{
	pdftitle={NIP'AJIN Flyer-Rollenspiel},
	pdfauthor={Markus Leupold-Loewenthal},
	pdfsubject={Ein NIP'AJIN-Rollenspiel auf einem Flyer. Setting und Regeln inklusive.},
	pdfkeywords={nipajin, nip'ajin, Rollenspiel, System, frei, RSP, RPG, Flyer}
}

\renewcommand{\flyerVersion}{v1.2.1}
\renewcommand{\flyerCredits}{%
	Text und Layout:~Markus Leupold-Löwenthal; Illustrationen:~Bettina Ott, all-silhouettes.com%
}

% --- back page ----------------------------------------------------------------

\renewcommand{\settingTitle}{\fontsize{12.5mm}{12mm}\selectfont

	\noindent\color{primary}\addfontfeature{LetterSpace=1.0}
		\parbox[b]{6.1cm}{\fontsize{11.0pt}{14pt}\selectfont
			„Eines Tages, vielleicht auch nie, werde \\ich dich bitten, diesen Flyer zu spielen.“
		}
	\color{white}\fffancy\addfontfeature{LetterSpace=2.0}
		FLYER

	\noindent\fffancy\addfontfeature{LetterSpace=12.0}
		ROLLENSPIEL!

}

\renewcommand{\settingHeadline}{Kleine Fische}

\renewcommand{\settingText}{%
	\noindent
	Dieses Szenario ist für 2 bis 6 Charaktere ausgelegt. Die Spielregeln findest du auf der anderen Seite.

	Einigt euch auf eine Stadt, einen kriminellen Hintergrund und eine Zeit, etwa die Mafia in New York um 1950 oder die Yakuza im heutigen Osaka. Die Charaktere sind dort konkurrierende \keyword{Kleinkriminelle}. Jeder Charakter erhält eine Beschreibung und einen Namen. Benutzt den Vierseiter als \HD. Ziel jedes Charakters ist es, die anderen zu überwinden, um selbst in die Gunst vom \keyword{Big Boss} -- dem Spielleiter -- zu kommen.

	Das Spiel wird in \keyword{Runden} von etwa einer Woche abgehandelt. Der Big Boss stellt der Gruppe jede Woche eine Aufgabe (Juwelenraub, Entführung,~\ldots), die ausgespielt wird, deren Erfolg aber nicht so wichtig ist. Innerhalb dieser Aufgabe darf jeder Charakter jede Woche einmalig versuchen, einem anderen zu schaden und dessen \HD\ um 1 zu senken. Überwundene Charaktere verlassen am Ende der Woche frustriert die Stadt.

	Ein Mal pro Spiel darf jeder, der sich ungerecht behandelt fühlt, zum Big Boss gehen und \keyword{um einen Gefallen bitten}. Gelingt der nötige Wurf (4+), lässt er alle verprügeln, die diese Woche den Charakter angegriffen haben (automatisch -2  am \HD). Misslingt der Wurf, wird jedoch der Charakter selbst bestraft und verprügelt.

}

\renewcommand{\settingURL}{\href{https://ludus-leonis.com/de/nipajin}{ludus-leonis.com}}
