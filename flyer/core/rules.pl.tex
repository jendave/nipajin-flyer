% (c) 2009-2019 by Markus Leupold-Löwenthal
% This file is released under CC BY-SA 4.0. Please do not apply one-way compatible licenses.

% CHANGELOG-pl
%
% v1.1.0
%   - start of changelog

\renewcommand{\rulesVersion}{v1.1.0}

% --- front page ---------------------------------------------------------------

\renewcommand{\rulesTitle}{

	\noindent\color{primary}\fontsize{10.5mm}{12mm}\selectfont\addfontfeature{WordSpace=2,LetterSpace=0.0}
	PRZEDSTAWIAMY WAM%

	\noindent\color{white}\fffancy\fontsize{10.5mm}{12mm}\selectfont\addfontfeature{WordSpace=2,LetterSpace=3}
	ERPEGOWĄ ULOTKĘ

	\vspace*{-1mm}

}

\renewcommand{\rulesHeadline}{Zasady gry}

\renewcommand{\rulesText}{%

	\noindent
	Ta ulotka zawiera skrót zasad \keyword{\nipajin} -- darmowej, uniwersalnej mechaniki RPG.

	Twoja \keyword{Postać} potrzebuje po jednej kości k4, k6, k8, k10 oraz k12. Wybierz, która z tych kości stanie się twoją \keyword{kością oporu} \HD\ i połóż ją tak, by jej najwyższy wynik był u góry. Jeśli wartość kości spadnie poniżej jedynki, Postać zostaje \keyword{pokonana} i wypada z gry. Reszta puli stanowi twoje \keyword{dostępne kości}.

	Jeżeli nie będziecie pewni, czy Postaci uda się jakaś czynność, wybierasz jedną \keyword{kość akcji} \AD\ spośród swoich dostępnych kości i rzucasz nią. Jeśli nikt nie przeszkadza, wystarczy wyrzucić \keyword{cztery lub więcej}, by osiągnąć sukces.

	Jeśli twoja czynność ma na celu skrzywdzenie przeciwnika, musi on wybrać swoją \keyword{kość reakcji} \RD. Jeśli twój wynik z \AD\ ma wyższą wartość niż \RD, atak kończy się sukcesem i wartość \HD\ przeciwnika spada o jeden.

	Wykorzystana kość zostaje \keyword{zużyta} i kładziesz ją na bok. Po zużyciu wszystkich kości odzyskujesz całą pulę.

	\bigskip

	\hfill Jasne? To teraz rozegraj scenariusz na drugiej stronie!
}

\renewcommand{\rulesURL}{\href{https://ludus-leonis.com/pl}{ludus-leonis.com/pl}}
