% (c) 2009-2019 by Markus Leupold-Löwenthal
% This file is released under CC BY-SA 4.0. Please do not apply one-way compatible licenses.

% CHANGELOG-de
%
% v1.1.1
%   - typo/wording fixes
% v1.1.0
%   - start of changelog

\renewcommand{\rulesVersion}{v1.1.1}

% --- front page ---------------------------------------------------------------

\renewcommand{\rulesTitle}{\fontsize{12.5mm}{12mm}\selectfont

	\noindent\color{primary}\addfontfeature{LetterSpace=-1.0}
		GESTATTEN,
	\color{white}\fffancy\addfontfeature{LetterSpace=2.0}
		FLYER

	\noindent\fffancy\addfontfeature{LetterSpace=12.0}
		ROLLENSPIEL!

}

\renewcommand{\rulesHeadline}{Spielregeln}

\renewcommand{\rulesText}{%

	\noindent
	Dieser Flyer enthält eine Kurzfassung von \keyword{\nipajin}, einem kostenlosen, universellen Rollenspielsystem.

	Dein \keyword{Charakter} benötigt je einen vier-, sechs-, acht-, zehn- und zwölfseitigen Würfel. Ernenne einen davon zum \keyword{Widerstandswürfel}~\HD. Lege\ ihn mit seiner höchsten Zahl auf den Tisch. Sinkt sein Wert im Spiel unter Eins, ist dein Charakter \keyword{überwunden} und scheidet aus. Die verbleibenden vier Würfel sind deine \keyword{verfügbaren Würfel}.

	Stellt sich die Frage, ob deinem Charakter eine Aktion gelingt, wählst du aus deinen verfügbaren Würfeln einen \keyword{Aktionswürfel}~\AD\ und würfelst diesen. Wehrt sich niemand, ist die Aktion bei einer \keyword{Vier oder mehr} ein Erfolg.

	Soll deine Aktion einem anderen Charakter schaden, muss dieser zur Verteidigung einen \keyword{Reaktionswürfel}~\RD\ aus seinen eigenen, verfügbaren Würfeln wählen. Nur wenn dein \AD\ einen höheren Wert als der \RD\ erwürfelt (unabhängig von einer Vier oder mehr), war dein Angriff ein Erfolg. Der \HD\ deines Gegners sinkt dann um einen Punkt.

	Nach jedem Wurf ist der benutzte Würfel \keyword{verbraucht}. Leg ihn beiseite. Erst wenn du alle deine Würfel verbraucht hast, werden sie wieder verfügbar.

	\hfill Spiele jetzt das Szenario auf der anderen Seite!

}

\renewcommand{\rulesURL}{\href{https://ludus-leonis.com/de/nipajin}{ludus-leonis.com}}
