% (c) 2019 by Markus Leupold-Löwenthal
% This file is released under CC BY-SA 4.0. Please do not apply one-way compatible licenses.

% CHANGELOG-de
%
% v1.0.1
%   - typo/wording fixes
% v1.0.0
%   - first version

% --- language dependent typography stuff --------------------------------------

\renewcommand{\say}[1]{„\textit{#1}“}
\setdefaultlanguage[spelling=new]{german}

\renewcommand{\fsFront}{\fontsize{11pt}{13.2pt plus 0pt minus 0pt}}
\renewcommand{\fsBack}{\fontsize{11pt}{13.2pt plus 0pt minus 0pt}}

% --- pdf metadata -------------------------------------------------------------

\hypersetup{
	pdftitle={NIP'AJIN Flyer-Rollenspiel},
	pdfauthor={Markus Leupold-Loewenthal},
	pdfsubject={Ein NIP'AJIN-Rollenspiel auf einem Flyer. Setting und Regeln inklusive.},
	pdfkeywords={nipajin, nip'ajin, Rollenspiel, System, frei, RSP, RPG, Flyer, Goblins}
}

\renewcommand{\flyerVersion}{v1.0.1}
\renewcommand{\flyerCredits}{%
	Text und Layout:~Markus Leupold-Löwenthal; Illustrationen:~Bettina Ott, Delapouite%
}

% --- back page ----------------------------------------------------------------

\renewcommand{\settingTitle}{\fontsize{12.5mm}{12mm}\selectfont

	\noindent\color{primary}\addfontfeature{LetterSpace=1.0}
		\parbox[b]{6.1cm}{\fontsize{11.0pt}{14pt}\selectfont
			Ein Szenario für 4 bis 6 Personen. Die Regeln findest du auf der Rückseite.
		}
	\color{white}\fffancy\addfontfeature{LetterSpace=2.0}
		FLYER

	\noindent\fffancy\addfontfeature{LetterSpace=12.0}
		ROLLENSPIEL!

}

\renewcommand{\settingHeadline}{Goblins}

\renewcommand{\settingText}{%
	\noindent
	Ihr seid \keyword{Goblins} -- kleine, grüne, chaotische Kreaturen mit hoher Stimme, einsilbigen Namen wie \say{Puk} oder \say{Mop} und dem Vierseiter als \HD. Euer Stamm feiert gerade in eurem Dorf den Geburtstag des Goblin-Königs, als sich dieser verschluckt, röchelt und tot zusammenbricht. Jeder seiner ehemaligen, engsten Vertrauten -- das seid ihr -- fühlt sich zur Nachfolge berufen. Rauft darum: Wer unüberwunden bleibt, wird neuer König!

	Das Spiel wird reihum in Runden abgehandelt. Bist du an der Reihe, erfindest du in der \keyword{ersten Runde} ein Detail der Goblins, des Dorfes oder des Festes. Etwa \say{Es brennt ein Freudenfeuer in der Dorfmitte.} oder \say{Der König starb an einem Stück Torte.} Dein Detail darf keinem anderen widersprechen und keinen Goblin ausgrenzen oder bevorzugen.

	Ab der \keyword{zweiten Runde} schilderst du auf Basis eines Details, das nicht von dir stammt, wie dein Goblin versucht, damit einem anderen Goblin zu schaden. Etwa \say{Puk versucht, Mop ins Feuer zu stoßen.} oder \say{Puk möchte Mop eine Kerze der Torte ins Auge rammen.} Lasst die Würfel entscheiden. Unabhängig von Erfolg oder Misserfolg legst du danach noch ein neues Detail fest.

	Goblins sind sehr neidisch. Sie greifen nur Gegner an, deren \HD-Wert gleich oder höher ist. Wenn sie alleine in Führung liegen, jubeln sie stolz, statt anzugreifen.
}

\renewcommand{\settingURL}{\href{https://ludus-leonis.com/de/nipajin}{ludus-leonis.com}}
